\documentclass[10pt,letterpaper]{article}
\usepackage[latin1]{inputenc}
\usepackage{amsmath}
\usepackage{amsfonts}
\usepackage{amssymb}
\usepackage{graphicx}
\usepackage{hyperref}
\oddsidemargin 0in
\evensidemargin 0in
\textwidth 6in
\textheight 8in
\author{T. Schneider \\ tes@mit.edu \\ gobysoft.com}
\title{Goby Underwater Autonomy Project Design Whitepaper}
\date{\today}
\begin{document}
\maketitle 

% views will be namespaces
% tables are different applications

\section{What is this?}
Goby is a new software infrastructure for autonomous vehicles that is intended to allow maximum interoperability in clusters of dissimilar AUVs and other marine vehicles. It is currently in the planning and initial development stage and feedback to the author would be greatly appreciated.

\section{Motivation}

Goby is motivated by the desire to create a versatile, robust, and highly extensible infrastructure for autonomous vehicles. The initial goal is to support underwater vehicles, so a good deal of emphasis is placed on minimizing data packet sizes for acoustic (and other low bandwidth) communications. The author has significant experience with the Mission Oriented Operating System (MOOS)\footnote{\url{http://www.robots.ox.ac.uk/~mobile/MOOS/wiki/pmwiki.php}} autonomy infrastructure which is quickly becoming a \textit{de facto} standard for robotic marine vehicles. Goby will have the ability to communicate fully with a MOOS community, allowing for full backwards compatibility and interoperability with MOOS.

Elements of Goby will be graciously borrowed from MOOS and LCM with a heavy emphasis on feasibility and scalability for marine robotics.

Seamless communication between vehicles (or subsystems of vehicles) connected by TCP/IP, UDP, acoustic, and serial links will be supported. Goby will run well on low power, embedded systems with the ability to suspend and resume at will.

Goby will be primarily geared at being a \textit{scientific} autonomy infrastructure. The driving features behind this goal are

\begin{itemize}
\item low power use for embedded systems on long duration deployment vehicles (e.g. gliders)
\item built in support for tracking accuracies and precisions of values within the system
\item robust support for runtime and postprocessing scientific software tools (MATLAB)
\end{itemize}

Goby will leverage well respected third party libraries\footnote{e.g. boost, asio, sqlite, dccl, Wt, protobuf} to increase quality and extensibility of the final project. 

\section{Overview}

\begin{itemize}
\item The infrastructure will be publish / subscribe based off a collection of individual processes. The author believes that this individual process model is an good way to handle development by a large number of contributors with differing levels of C++ skill. Individual processes will pass messages using Google Protocol Buffers (protobuf) through shared memory message queues (boost-interprocess).
\item The core process (\verb|gobyd|) will write all transactions to an SQL backend (choice of SQLite\footnote{\url{http://www.sqlite.org/}}, MySQL, PostgreSQL database), unlike the MOOSDB which is not actually a database, but simply a message handler. SQLite uses single files for entire databases, which means the log at the end of a run will still be a single file. This base database structure also removes the need for separate "logging" processes. The database interface will be managed using the objects-based Wt-Dbo library to give a seamless interface between protobuf classes and the database.
\item \verb|gobyd| will be a daemon that can spawn, exit, kill the other processes running.
\item Goby will distinguish from the local machine and ``everyone else''. Processes running on the local community will communicate via shared memory (boost::interprocess) and (optionally) directly read the SQL database (via the filesystem). Communities running elsewhere will be linked via the \verb|goby_core| using available communications (wireless ethernet, serial, acoustic communications). This allows for very high speed transfers internally, which is sacrificed when everything is a TCP/IP connection. 
\item Processes will be configured using XML that will be read into a protobuf object (MyProcessConfig).
\item \verb|gobyd| will have be a Wt\footnote{\url{http://www.webtoolkit.eu/wt}} enabled webserver which will form the interface (via any Javascript enabled browser). From here, configuration, runtime monitoring, commanding, etc. will be performed. The user simply connects to http://\{machine\}:\{goby-port\} to administer it. 
\end{itemize}

\section{Features}

\begin{itemize}
\item Protobuf Structures
\begin{itemize}
\item Problem: A running AUV community with 30 processes because an incoherent sea of variables. We need some way to group them in a coherent manner. Designers need flexibility to user a variety of standard types to form these messages.
\item Solution: Use protobuf to group messages into sensible structures.
\end{itemize}
\item Mandatory variables
\begin{itemize}
\item Problem: it is often unclear whether all preconditions for success are verified at launch time.
\item Solution: certain variables are critical to the success of the vehicle's mission. If these are not being published a warning must be posted. For example, most of the nav namespace would be mandatory.
\end{itemize}
\item Seamless integration with acoustic networking software (DCCL, queue, amac, modemdriver)
\item Cold power off and resume.
\end{itemize}

The basic message sent will be a GobyMessage (which is a protobuf message) and will contain the following fields:
\begin{itemize}
\item Timestamp of generation (UNIX seconds / fractional seconds since 1970)
\item Source (if local, what process wrote this; if from outside, what community wrote this).
\item Destination (e.g. \textit{local}, \textit{all}, or \textit{specific vehicle})
\item Time To Live (in seconds)
\item Priority (very low, low, medium, high, very high)
\item Acknowledgement requested (only for \textit{specific vehicle}, \textit{all} has no guarantees, and \textit{local} is always guaranteed).
\end{itemize}

Users are encouraged to extend GobyMessage with the fields they need for whatever data is to be sent.

\section{Requirements of Applications Authors}

Basic:
\begin{itemize}
\item Overloaded GobyApplication:
\begin{itemize}
\item \texttt{Constructor}: read configuration, subscribe for variables, initialize data, etc.
\item \texttt{loop}: called at some predefined frequency (say 10 Hz), if overloaded.
\item \texttt{inbox}: called for messages for which a custom handlers are not provided. Because of protobuf, you will need to use reflection so this method may be scratched or relegated to the ``advanced'' bin.
\item custom asynchronous event-driven handlers (i.e. some method or function) for different types of mail requested (a handler for each protobuf type). Generalization of \texttt{inbox}.
\item custom synchronous handlers (i.e. some method or function) called on some timer. Generalization of \texttt{loop}.
\end{itemize}
\item publish(name, value, dest = local, ttl = 1800, priority = medium, type = ): call to send mail.
\end{itemize}

Advanced:
\begin{itemize}
\item Overloaded GobyApplication:
\begin{itemize}
\item Basic methods
\item ack: called when an acknowledgement for data sent over the wire arrives
\end{itemize}
\end{itemize}



\section{Licensing}
Goby will be free open source software licensed under the GNU General Public License v3 (GPLv3)\footnote{\url{http://www.gnu.org/licenses/}}. The philosophy of using a strong copyleft license can be summed up with ``what starts free should remain free''. In the spirit of free scientific exchange, this software will be available to be improved by all and will remain that way.

\section{Compatibility with MOOS}
Goby will provide a mimic of CMOOSApp that extends GobyApplication providing the same interface. This will allow for graceful migration of processes to the richer feature set of Goby without modifying much or any code.

\end{document}